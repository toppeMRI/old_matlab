
\chapter{Using toppe.e as an interpreter module for Pulseq files}
\label{ch:pulseq}

\section{Pulseq}
An effort is currently underway to make\toppe~compatible with Pulseq, an open file format for compactly describing MR sequences.
The Pulseq file specification, along with supporting Matlab and C\texttt{++} libraries, is available at
\begin{center}
\url{http://pulseq.github.io/}
\end{center}
Pulseq relies on vendor-dependent ``interpreter modules'' to load a Pulseq (.seq) file onto a particular scanner platform.
\toppe~can serve as the interpreter module for GE scanners.
Interpreters currently also exist for Siemens and Bruker scanners, enabling truly platform-independent MR pulse programming.
The following publication has more information about the Pulseq platform and philosophy:

\vspace{10pt}
{\it \small Layton K, Kroboth S, Jia F, Littin S, Yu H, Leupold J, Nielsen JF, St{\"o}cker T, Zaitsev M. Pulseq: A rapid and hardware-independent pulse sequence prototyping framework. Magn Reson Med 2016 (to appear).}
\vspace{10pt}


\section{Using \toppe~to play .seq files}
To use\toppe~as a GE interpreter module for Pulseq files, use the Matlab script {\tt \bf seq2ge.m} in the {\tt pulseq} directory in this distribution.
{\tt seq2ge.m} takes as input a .seq file and outputs the various files needed by\toppe~({\tt cores.txt}, {\tt scanloop.txt}, and .wav files).
For an example, see \texttt{main.m} in the \texttt{pulseq} directory.
%{\tt seq2ge.m} has so far been tested only with a simple 2D gradient-echo sequence ({\tt gre.seq}been does not support all Pulseq features, and we will have more to say here in the future as the script is updated.

%\section{Tools for Pulseq file creation}
%Pulseq 
